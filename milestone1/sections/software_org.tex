\subsection{Directory Structure}
 {\color{blue} /Ccs207-FinalProject}\\
    \hspace*{1cm}{\color{blue}/src}  \hfill Back-end source code\\
		\hspace*{1cm}\hspace*{1cm}{\color{blue}/config} \hfill Configuration for the project\\
		\hspace*{1cm}\hspace*{1cm}{\color{blue}Auto\_diff.py} \\
		\hspace*{1cm}\hspace*{1cm}{\color{blue}forward.py} \\
		\hspace*{1cm}\hspace*{1cm}{\color{blue}reverse.py} \\
		\hspace*{1cm}\hspace*{1cm}{\color{blue}basic\_funcs.py (add by Xin)} \\
		\hspace*{1cm}\hspace*{1cm}{\color{blue}...} \\
	\hspace*{1cm}{\color{blue}/gui}  \hfill Front-end source code \\
		\hspace*{1cm}\hspace*{1cm}{\color{blue}/dist}\hfill Static css, js etc.\\
		\hspace*{1cm}\hspace*{1cm}{\color{blue}/template}\hfill Web html files\\
		\hspace*{1cm}\hspace*{1cm}{\color{blue}/img}\hfill Images used for font-end\\
	\hspace*{1cm}{\color{blue}/utils}  \hfill Preprocessing scripts \\
	\hspace*{1cm}\hspace*{1cm}{\color{blue}input\_parser.py}\\
	\hspace*{1cm}\hspace*{1cm}{\color{blue}...}\\
	\hspace*{1cm}{\color{blue}/test} \hfill Test cases \\
	\hspace*{1cm}\hspace*{1cm}{\color{blue}test\_forward.py}\\
	\hspace*{1cm}\hspace*{1cm}{\color{blue}test\_reverse.py}\\
	\hspace*{1cm}\hspace*{1cm}{\color{blue}test\_eval.py}\\
	\hspace*{1cm}\hspace*{1cm}{\color{blue}...}\\
    \hspace*{1cm}{\color{blue}/doc} \hfill Documentation and records\\
          \hspace*{1cm}\hspace*{1cm}{\color{blue}milestone1.tex}\\
          \hspace*{1cm}\hspace*{1cm}{\color{blue}milestone2.tex}\\
          \hspace*{1cm}\hspace*{1cm}{\color{blue}...}\\
    \hspace*{1cm}{\color{blue}\_\_init\_\_.py} \hfill Initialization\\
    \hspace*{1cm}{\color{blue}requirements.txt}\hfill Packages on which the program depends\\
	\hspace*{1cm}{\color{blue}README.md} \hfill Introduction for the project
\subsection{Modules and functionality}
We plan to include: 
\begin{itemize}
    \item AutoDiff module for definition of the AutoDiff class.
    \item Forward module for forward mode in automatic differentiation.
    \item Reverse module for reverse mode in automatic differentiation.
    \item Some Utils modules for parsing the input, preprocessing and start main program.
\end{itemize}
 
\subsection{Test Suite}
Coding is the fundamental part of software development. Equally significant is build and testing. We would utilize \emph{Travis CI} and \emph{CodeCov} to make the development process more reliable and professional. The test suite will be placed in the test folder.
\begin{itemize}
    \item \emph{Travis CI} is used as a distributed CI (Continuous Integration) tools to build and automate test the project.
    \item \emph{CodeCov} is used for test results analysis (eg. measuring test code coverage) and visualization.
\end{itemize}
 

\subsection{Software Package and Distribution}
\begin{itemize}
    \item  Package distribution\\
    We will package our software using PyPI (Python Package Index) for release. 
    Write and run 'setup.py' to package the software and upload it to the distribution server, thus people in community could easily download our package by 'pip install'.
    \item Version Control\\
    We will take Version Control into consideration according to the standard in Python Enhancement Proposal (PEP) 386. With version control, we can tell the user what changes we made and set clear boundaries for where those changes occurred.
    \item Framework
    \begin{itemize}
        \item For web development, we would use \emph{Flask}, a micro web framework, which is suitable for a small team to complete the implementation of a feature-rich small website and easily add customized functions.
        \item For GUI (Graphical User Interface), we may choose Vue.js, a JavaScript framework for building user interfaces and single-page applications. Because it offers many API (Application Program Interface) to integrate with existing projects and is easy to get started. It is better in code reuse compared to frameworks like jQuery.

    \end{itemize}
    
    
\end{itemize}





