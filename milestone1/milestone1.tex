\input{/home/eruiz/dev/latex/preamble.tex}
\lstset{language=Python}
\begin{document}
    \title{CS 207: Milestone 1}
    \author{Erick Ruiz\\ \texttt{eruiz@g.harvard.edu}}
    \date{October 26, 2019}
    \maketitle
    
    \section*{Introduction}
    Understanding the concept of a derivative is crucial to all aspiring and 
    practicing scientists, engineers, and mathematicians. It is one of the first
    concepts introduced in first-year calculus courses at all universities. The 
    idea is simple. Given a function, $f(x)$, how can we quantify the rate of 
    change of the function due to an infinitesimal change, $\Delta x$, in the 
    argument, $x$? The answer is typically given in terms of the limit 
    definition of the derivative.
    \begin{equation}
        f'(x) = \lim_{\Delta x\rightarrow 0} \frac{f(x+\Delta x)-f(x)}{\Delta x}
        \label{eq:lim-def}
    \end{equation}
    While equation \eqref{eq:lim-def} holds for any function, in practice, it is
    easier to calculate derivatives analytically according to a set of rules. 
    However, obtaining an analytical expression for the derivative becomes 
    exceedingly difficult if the function of interest is composed of many 
    elementary functions. For example, consider the following function.
    \begin{equation}
        f(x) = \exp\left[\frac{\sqrt{x^3 - \ln x + \sin(4x^2)}}{\cos(3x^5)}\right]
        \label{eq:ugly-eq}
    \end{equation}
    Calculating the first derivative would result in the following expression.
    \begin{align}
        f'(x) &= \exp\left[\frac{\sqrt{x^3 - \ln x + \sin(4x^2)}}{\cos(3x^5)}\right]
        \sec^2(3x^5)\dots\nonumber\\
        &\qquad\times\left\{
        \frac{\cos(3x^5)}{2\sqrt{x^3 - \ln x + \sin(4x^2)}}
        \left[3x^2-\frac{1}{x}+8x\cos(4x^2)\right]\dots\right.\nonumber\\
        &\qquad\qquad+
        \left.\vphantom{\frac{\cos(3x^5)}{2\sqrt{x^3 - \ln x + \sin(4x^2)}}}
        15x^4\sin(3x^5)\sqrt{x^3 - \ln x + \sin(4x^2)}
        \right\}
    \end{align}
    Although feasible, successive calculations become more and more complex, and
    in practice, the quantity to be differentiated may not be a function in 
    closed-form but rather a set of measurements or values given as a 
    one-dimensional vector of numbers. In that case, equation \eqref{eq:lim-def}
    can be approximated using the finite difference method, which replaces an 
    infinitesimal change in the argument for a finite change. To show how this 
    works, let us write the Taylor series expansion of an arbitrary function, 
    $f(x)$, at the point $x+h$.
    \begin{equation}
        f(x+h) = f(x) + hf'(x) + \frac{h^2}{2}f''(x) + \dots
        \label{eq:f-tay}
    \end{equation}
    Keeping only terms of $\mathcal{O}(h)$ leaves us with
    \begin{equation}
        f(x) \approx f(h) + hf'(x),
    \end{equation}
    which we can rearrange to write an approximate expression for the derivative,
    $f'(x)$.
    \begin{equation}
        f'(x) \approx \frac{f(x+h)-f(x)}{h}
        \label{eq:fd}
    \end{equation}
    The finite change, $h$, is called the step size, and equation \eqref{eq:fd} 
    is known as the forward difference. 
    \begin{figure}[H]
        \centering
        \begin{tikzpicture}
            % Help lines
            % \draw[help lines] (0,0) grid (6,4);
            
            % Axes
            \draw[<->, >=stealth, thick] (0,4) node[above]{$y$} -- (0,0) -- 
            (6,0) node[right]{$x$};
            
            % Some function
            \draw[smooth, samples=100, domain=0.0:4.85] 
                plot(\x,{0.1*\x^2-1.1*sin(\x)*\x^3+exp(0.1*\x)*\x});
                
            % Tangent line
            \draw[red] (3,3.4) -- (4,2.65);
            
            % Step size
            \draw[dashed] (3,0) -- (3,3.4);
            \draw[dashed] (4,0) -- (4,2.65);
            \draw[|<->|, >=stealth] (3,-0.3) -- (3.5,-0.3) node[fill=white]{$h$}
                -- (4,-0.3);
            
            % Function annotations
            \draw[dashed] (3,3.4) -- (0,3.4) node[left]{$f(x)$};
            \draw[dashed] (4,2.65) -- (0,2.65) node[left]{$f(x+h)$};
        \end{tikzpicture}
        \caption{\textbf{Geometric interpretation of the finite difference 
        method.} As the step size, $h$, decreases, so does the difference 
        between $f(x+h)$ and $f(x)$. In principle, this should lead to a more 
        accurate approximation to the true derivative, $f'(x)$. However, this is
        not always the case.}
        \label{fig:fd-schematic}
    \end{figure}
    Although the finite difference method is useful and easy to implement, its 
    accuracy can vary depending on the step size that is chosen. Suppose we wish
    to approximate the derivative of $f(x)=\ln x$ using the forward difference 
    method described in equation \eqref{eq:fd} using step sizes $h=\{10^{-1},\,
    10^{-7},\,10^{-15}\}$. This is rather unnecessary because the analytical 
    derivative is just $f'(x) = 1/x$, but this example will serve to illustrate 
    the drawbacks of the finite difference method. At $h=10^{-1}$, the numerical
    derivative is inaccurate because the step size is too large, making the 
    calculations susceptible to truncation error. Conversely, at $h=10^{-15}$, 
    the forward difference method also gives inaccurate results because the 
    calculations can only be represented to a finite precision by the hardware 
    in use. Hence, rounding error also affects the stability of the finite 
    difference method. Figure \ref{fig:fd-accuracy} summarizes the results.
    % Finite difference accuracy
    % --------------------------
    % Borrowed from CS 207: Homework 4, Problem 1 
    \begin{figure}
        \centering
        \begin{tikzpicture}[gnuplot]
%% generated with GNUPLOT 5.2p6 (Lua 5.3; terminal rev. Nov 2018, script rev. 107)
%% Sun 27 Oct 2019 01:05:40 PM EDT
\path (0.000,0.000) rectangle (12.500,8.750);
\gpcolor{color=gp lt color axes}
\gpsetlinetype{gp lt axes}
\gpsetdashtype{gp dt axes}
\gpsetlinewidth{0.50}
\draw[gp path] (1.320,0.985)--(11.947,0.985);
\gpcolor{color=gp lt color border}
\gpsetlinetype{gp lt border}
\gpsetdashtype{gp dt solid}
\gpsetlinewidth{1.00}
\draw[gp path] (1.320,0.985)--(1.500,0.985);
\draw[gp path] (11.947,0.985)--(11.767,0.985);
\node[gp node right] at (1.136,0.985) {$2$};
\gpcolor{color=gp lt color axes}
\gpsetlinetype{gp lt axes}
\gpsetdashtype{gp dt axes}
\gpsetlinewidth{0.50}
\draw[gp path] (1.320,2.228)--(11.947,2.228);
\gpcolor{color=gp lt color border}
\gpsetlinetype{gp lt border}
\gpsetdashtype{gp dt solid}
\gpsetlinewidth{1.00}
\draw[gp path] (1.320,2.228)--(1.500,2.228);
\draw[gp path] (11.947,2.228)--(11.767,2.228);
\node[gp node right] at (1.136,2.228) {$2.5$};
\gpcolor{color=gp lt color axes}
\gpsetlinetype{gp lt axes}
\gpsetdashtype{gp dt axes}
\gpsetlinewidth{0.50}
\draw[gp path] (1.320,3.470)--(11.947,3.470);
\gpcolor{color=gp lt color border}
\gpsetlinetype{gp lt border}
\gpsetdashtype{gp dt solid}
\gpsetlinewidth{1.00}
\draw[gp path] (1.320,3.470)--(1.500,3.470);
\draw[gp path] (11.947,3.470)--(11.767,3.470);
\node[gp node right] at (1.136,3.470) {$3$};
\gpcolor{color=gp lt color axes}
\gpsetlinetype{gp lt axes}
\gpsetdashtype{gp dt axes}
\gpsetlinewidth{0.50}
\draw[gp path] (1.320,4.713)--(11.947,4.713);
\gpcolor{color=gp lt color border}
\gpsetlinetype{gp lt border}
\gpsetdashtype{gp dt solid}
\gpsetlinewidth{1.00}
\draw[gp path] (1.320,4.713)--(1.500,4.713);
\draw[gp path] (11.947,4.713)--(11.767,4.713);
\node[gp node right] at (1.136,4.713) {$3.5$};
\gpcolor{color=gp lt color axes}
\gpsetlinetype{gp lt axes}
\gpsetdashtype{gp dt axes}
\gpsetlinewidth{0.50}
\draw[gp path] (1.320,5.956)--(11.947,5.956);
\gpcolor{color=gp lt color border}
\gpsetlinetype{gp lt border}
\gpsetdashtype{gp dt solid}
\gpsetlinewidth{1.00}
\draw[gp path] (1.320,5.956)--(1.500,5.956);
\draw[gp path] (11.947,5.956)--(11.767,5.956);
\node[gp node right] at (1.136,5.956) {$4$};
\gpcolor{color=gp lt color axes}
\gpsetlinetype{gp lt axes}
\gpsetdashtype{gp dt axes}
\gpsetlinewidth{0.50}
\draw[gp path] (1.320,7.198)--(8.777,7.198);
\draw[gp path] (11.763,7.198)--(11.947,7.198);
\gpcolor{color=gp lt color border}
\gpsetlinetype{gp lt border}
\gpsetdashtype{gp dt solid}
\gpsetlinewidth{1.00}
\draw[gp path] (1.320,7.198)--(1.500,7.198);
\draw[gp path] (11.947,7.198)--(11.767,7.198);
\node[gp node right] at (1.136,7.198) {$4.5$};
\gpcolor{color=gp lt color axes}
\gpsetlinetype{gp lt axes}
\gpsetdashtype{gp dt axes}
\gpsetlinewidth{0.50}
\draw[gp path] (1.320,8.441)--(11.947,8.441);
\gpcolor{color=gp lt color border}
\gpsetlinetype{gp lt border}
\gpsetdashtype{gp dt solid}
\gpsetlinewidth{1.00}
\draw[gp path] (1.320,8.441)--(1.500,8.441);
\draw[gp path] (11.947,8.441)--(11.767,8.441);
\node[gp node right] at (1.136,8.441) {$5$};
\gpcolor{color=gp lt color axes}
\gpsetlinetype{gp lt axes}
\gpsetdashtype{gp dt axes}
\gpsetlinewidth{0.50}
\draw[gp path] (1.320,0.985)--(1.320,8.441);
\gpcolor{color=gp lt color border}
\gpsetlinetype{gp lt border}
\gpsetdashtype{gp dt solid}
\gpsetlinewidth{1.00}
\draw[gp path] (1.320,0.985)--(1.320,1.165);
\draw[gp path] (1.320,8.441)--(1.320,8.261);
\node[gp node center] at (1.320,0.677) {$0.2$};
\gpcolor{color=gp lt color axes}
\gpsetlinetype{gp lt axes}
\gpsetdashtype{gp dt axes}
\gpsetlinewidth{0.50}
\draw[gp path] (3.977,0.985)--(3.977,8.441);
\gpcolor{color=gp lt color border}
\gpsetlinetype{gp lt border}
\gpsetdashtype{gp dt solid}
\gpsetlinewidth{1.00}
\draw[gp path] (3.977,0.985)--(3.977,1.165);
\draw[gp path] (3.977,8.441)--(3.977,8.261);
\node[gp node center] at (3.977,0.677) {$0.25$};
\gpcolor{color=gp lt color axes}
\gpsetlinetype{gp lt axes}
\gpsetdashtype{gp dt axes}
\gpsetlinewidth{0.50}
\draw[gp path] (6.633,0.985)--(6.633,8.441);
\gpcolor{color=gp lt color border}
\gpsetlinetype{gp lt border}
\gpsetdashtype{gp dt solid}
\gpsetlinewidth{1.00}
\draw[gp path] (6.633,0.985)--(6.633,1.165);
\draw[gp path] (6.633,8.441)--(6.633,8.261);
\node[gp node center] at (6.633,0.677) {$0.3$};
\gpcolor{color=gp lt color axes}
\gpsetlinetype{gp lt axes}
\gpsetdashtype{gp dt axes}
\gpsetlinewidth{0.50}
\draw[gp path] (9.290,0.985)--(9.290,6.413);
\draw[gp path] (9.290,8.261)--(9.290,8.441);
\gpcolor{color=gp lt color border}
\gpsetlinetype{gp lt border}
\gpsetdashtype{gp dt solid}
\gpsetlinewidth{1.00}
\draw[gp path] (9.290,0.985)--(9.290,1.165);
\draw[gp path] (9.290,8.441)--(9.290,8.261);
\node[gp node center] at (9.290,0.677) {$0.35$};
\gpcolor{color=gp lt color axes}
\gpsetlinetype{gp lt axes}
\gpsetdashtype{gp dt axes}
\gpsetlinewidth{0.50}
\draw[gp path] (11.947,0.985)--(11.947,8.441);
\gpcolor{color=gp lt color border}
\gpsetlinetype{gp lt border}
\gpsetdashtype{gp dt solid}
\gpsetlinewidth{1.00}
\draw[gp path] (11.947,0.985)--(11.947,1.165);
\draw[gp path] (11.947,8.441)--(11.947,8.261);
\node[gp node center] at (11.947,0.677) {$0.4$};
\draw[gp path] (1.320,8.441)--(1.320,0.985)--(11.947,0.985)--(11.947,8.441)--cycle;
\node[gp node center,rotate=-270] at (0.276,4.713) {$y$};
\node[gp node center] at (6.633,0.215) {$x$};
\node[gp node left] at (8.777,8.030) {$f'(x)$};
\gpcolor{rgb color={0.580,0.000,0.827}}
\gpsetlinewidth{4.00}
\draw[gp path] (10.663,8.030)--(11.579,8.030);
\draw[gp path] (1.320,8.441)--(1.427,8.317)--(1.535,8.195)--(1.642,8.076)--(1.749,7.958)%
  --(1.857,7.844)--(1.964,7.731)--(2.071,7.620)--(2.179,7.512)--(2.286,7.405)--(2.393,7.301)%
  --(2.501,7.198)--(2.608,7.098)--(2.715,6.999)--(2.823,6.901)--(2.930,6.806)--(3.037,6.712)%
  --(3.145,6.620)--(3.252,6.529)--(3.360,6.440)--(3.467,6.352)--(3.574,6.266)--(3.682,6.182)%
  --(3.789,6.098)--(3.896,6.016)--(4.004,5.936)--(4.111,5.856)--(4.218,5.778)--(4.326,5.701)%
  --(4.433,5.626)--(4.540,5.551)--(4.648,5.478)--(4.755,5.405)--(4.862,5.334)--(4.970,5.264)%
  --(5.077,5.195)--(5.184,5.127)--(5.292,5.060)--(5.399,4.994)--(5.506,4.929)--(5.614,4.865)%
  --(5.721,4.802)--(5.828,4.739)--(5.936,4.678)--(6.043,4.617)--(6.150,4.558)--(6.258,4.499)%
  --(6.365,4.441)--(6.472,4.383)--(6.580,4.327)--(6.687,4.271)--(6.795,4.216)--(6.902,4.162)%
  --(7.009,4.108)--(7.117,4.055)--(7.224,4.003)--(7.331,3.951)--(7.439,3.900)--(7.546,3.850)%
  --(7.653,3.801)--(7.761,3.752)--(7.868,3.703)--(7.975,3.656)--(8.083,3.608)--(8.190,3.562)%
  --(8.297,3.516)--(8.405,3.470)--(8.512,3.425)--(8.619,3.381)--(8.727,3.337)--(8.834,3.294)%
  --(8.941,3.251)--(9.049,3.209)--(9.156,3.167)--(9.263,3.126)--(9.371,3.085)--(9.478,3.044)%
  --(9.585,3.004)--(9.693,2.965)--(9.800,2.926)--(9.907,2.887)--(10.015,2.849)--(10.122,2.811)%
  --(10.230,2.774)--(10.337,2.737)--(10.444,2.700)--(10.552,2.664)--(10.659,2.629)--(10.766,2.593)%
  --(10.874,2.558)--(10.981,2.524)--(11.088,2.489)--(11.196,2.455)--(11.303,2.422)--(11.410,2.389)%
  --(11.518,2.356)--(11.625,2.323)--(11.732,2.291)--(11.840,2.259)--(11.947,2.228);
\gpcolor{color=gp lt color border}
\node[gp node left] at (8.777,7.568) {$h=10^{-1}$};
\gpcolor{rgb color={0.000,0.620,0.451}}
\gpsetlinewidth{1.00}
\draw[gp path] (10.663,7.568)--(11.579,7.568);
\draw[gp path] (1.320,6.091)--(1.427,6.009)--(1.535,5.927)--(1.642,5.847)--(1.749,5.768)%
  --(1.857,5.690)--(1.964,5.614)--(2.071,5.538)--(2.179,5.464)--(2.286,5.391)--(2.393,5.320)%
  --(2.501,5.249)--(2.608,5.179)--(2.715,5.111)--(2.823,5.043)--(2.930,4.977)--(3.037,4.911)%
  --(3.145,4.847)--(3.252,4.783)--(3.360,4.720)--(3.467,4.659)--(3.574,4.598)--(3.682,4.538)%
  --(3.789,4.478)--(3.896,4.420)--(4.004,4.362)--(4.111,4.306)--(4.218,4.250)--(4.326,4.194)%
  --(4.433,4.140)--(4.540,4.086)--(4.648,4.033)--(4.755,3.981)--(4.862,3.929)--(4.970,3.878)%
  --(5.077,3.828)--(5.184,3.778)--(5.292,3.729)--(5.399,3.680)--(5.506,3.633)--(5.614,3.585)%
  --(5.721,3.539)--(5.828,3.493)--(5.936,3.447)--(6.043,3.402)--(6.150,3.358)--(6.258,3.314)%
  --(6.365,3.270)--(6.472,3.228)--(6.580,3.185)--(6.687,3.143)--(6.795,3.102)--(6.902,3.061)%
  --(7.009,3.021)--(7.117,2.981)--(7.224,2.941)--(7.331,2.902)--(7.439,2.864)--(7.546,2.826)%
  --(7.653,2.788)--(7.761,2.750)--(7.868,2.714)--(7.975,2.677)--(8.083,2.641)--(8.190,2.605)%
  --(8.297,2.570)--(8.405,2.535)--(8.512,2.500)--(8.619,2.466)--(8.727,2.432)--(8.834,2.399)%
  --(8.941,2.366)--(9.049,2.333)--(9.156,2.300)--(9.263,2.268)--(9.371,2.237)--(9.478,2.205)%
  --(9.585,2.174)--(9.693,2.143)--(9.800,2.113)--(9.907,2.082)--(10.015,2.052)--(10.122,2.023)%
  --(10.230,1.993)--(10.337,1.964)--(10.444,1.936)--(10.552,1.907)--(10.659,1.879)--(10.766,1.851)%
  --(10.874,1.823)--(10.981,1.796)--(11.088,1.769)--(11.196,1.742)--(11.303,1.715)--(11.410,1.689)%
  --(11.518,1.662)--(11.625,1.637)--(11.732,1.611)--(11.840,1.585)--(11.947,1.560);
\gpsetpointsize{4.00}
\gppoint{gp mark 2}{(1.320,6.091)}
\gppoint{gp mark 2}{(1.427,6.009)}
\gppoint{gp mark 2}{(1.535,5.927)}
\gppoint{gp mark 2}{(1.642,5.847)}
\gppoint{gp mark 2}{(1.749,5.768)}
\gppoint{gp mark 2}{(1.857,5.690)}
\gppoint{gp mark 2}{(1.964,5.614)}
\gppoint{gp mark 2}{(2.071,5.538)}
\gppoint{gp mark 2}{(2.179,5.464)}
\gppoint{gp mark 2}{(2.286,5.391)}
\gppoint{gp mark 2}{(2.393,5.320)}
\gppoint{gp mark 2}{(2.501,5.249)}
\gppoint{gp mark 2}{(2.608,5.179)}
\gppoint{gp mark 2}{(2.715,5.111)}
\gppoint{gp mark 2}{(2.823,5.043)}
\gppoint{gp mark 2}{(2.930,4.977)}
\gppoint{gp mark 2}{(3.037,4.911)}
\gppoint{gp mark 2}{(3.145,4.847)}
\gppoint{gp mark 2}{(3.252,4.783)}
\gppoint{gp mark 2}{(3.360,4.720)}
\gppoint{gp mark 2}{(3.467,4.659)}
\gppoint{gp mark 2}{(3.574,4.598)}
\gppoint{gp mark 2}{(3.682,4.538)}
\gppoint{gp mark 2}{(3.789,4.478)}
\gppoint{gp mark 2}{(3.896,4.420)}
\gppoint{gp mark 2}{(4.004,4.362)}
\gppoint{gp mark 2}{(4.111,4.306)}
\gppoint{gp mark 2}{(4.218,4.250)}
\gppoint{gp mark 2}{(4.326,4.194)}
\gppoint{gp mark 2}{(4.433,4.140)}
\gppoint{gp mark 2}{(4.540,4.086)}
\gppoint{gp mark 2}{(4.648,4.033)}
\gppoint{gp mark 2}{(4.755,3.981)}
\gppoint{gp mark 2}{(4.862,3.929)}
\gppoint{gp mark 2}{(4.970,3.878)}
\gppoint{gp mark 2}{(5.077,3.828)}
\gppoint{gp mark 2}{(5.184,3.778)}
\gppoint{gp mark 2}{(5.292,3.729)}
\gppoint{gp mark 2}{(5.399,3.680)}
\gppoint{gp mark 2}{(5.506,3.633)}
\gppoint{gp mark 2}{(5.614,3.585)}
\gppoint{gp mark 2}{(5.721,3.539)}
\gppoint{gp mark 2}{(5.828,3.493)}
\gppoint{gp mark 2}{(5.936,3.447)}
\gppoint{gp mark 2}{(6.043,3.402)}
\gppoint{gp mark 2}{(6.150,3.358)}
\gppoint{gp mark 2}{(6.258,3.314)}
\gppoint{gp mark 2}{(6.365,3.270)}
\gppoint{gp mark 2}{(6.472,3.228)}
\gppoint{gp mark 2}{(6.580,3.185)}
\gppoint{gp mark 2}{(6.687,3.143)}
\gppoint{gp mark 2}{(6.795,3.102)}
\gppoint{gp mark 2}{(6.902,3.061)}
\gppoint{gp mark 2}{(7.009,3.021)}
\gppoint{gp mark 2}{(7.117,2.981)}
\gppoint{gp mark 2}{(7.224,2.941)}
\gppoint{gp mark 2}{(7.331,2.902)}
\gppoint{gp mark 2}{(7.439,2.864)}
\gppoint{gp mark 2}{(7.546,2.826)}
\gppoint{gp mark 2}{(7.653,2.788)}
\gppoint{gp mark 2}{(7.761,2.750)}
\gppoint{gp mark 2}{(7.868,2.714)}
\gppoint{gp mark 2}{(7.975,2.677)}
\gppoint{gp mark 2}{(8.083,2.641)}
\gppoint{gp mark 2}{(8.190,2.605)}
\gppoint{gp mark 2}{(8.297,2.570)}
\gppoint{gp mark 2}{(8.405,2.535)}
\gppoint{gp mark 2}{(8.512,2.500)}
\gppoint{gp mark 2}{(8.619,2.466)}
\gppoint{gp mark 2}{(8.727,2.432)}
\gppoint{gp mark 2}{(8.834,2.399)}
\gppoint{gp mark 2}{(8.941,2.366)}
\gppoint{gp mark 2}{(9.049,2.333)}
\gppoint{gp mark 2}{(9.156,2.300)}
\gppoint{gp mark 2}{(9.263,2.268)}
\gppoint{gp mark 2}{(9.371,2.237)}
\gppoint{gp mark 2}{(9.478,2.205)}
\gppoint{gp mark 2}{(9.585,2.174)}
\gppoint{gp mark 2}{(9.693,2.143)}
\gppoint{gp mark 2}{(9.800,2.113)}
\gppoint{gp mark 2}{(9.907,2.082)}
\gppoint{gp mark 2}{(10.015,2.052)}
\gppoint{gp mark 2}{(10.122,2.023)}
\gppoint{gp mark 2}{(10.230,1.993)}
\gppoint{gp mark 2}{(10.337,1.964)}
\gppoint{gp mark 2}{(10.444,1.936)}
\gppoint{gp mark 2}{(10.552,1.907)}
\gppoint{gp mark 2}{(10.659,1.879)}
\gppoint{gp mark 2}{(10.766,1.851)}
\gppoint{gp mark 2}{(10.874,1.823)}
\gppoint{gp mark 2}{(10.981,1.796)}
\gppoint{gp mark 2}{(11.088,1.769)}
\gppoint{gp mark 2}{(11.196,1.742)}
\gppoint{gp mark 2}{(11.303,1.715)}
\gppoint{gp mark 2}{(11.410,1.689)}
\gppoint{gp mark 2}{(11.518,1.662)}
\gppoint{gp mark 2}{(11.625,1.637)}
\gppoint{gp mark 2}{(11.732,1.611)}
\gppoint{gp mark 2}{(11.840,1.585)}
\gppoint{gp mark 2}{(11.947,1.560)}
\gppoint{gp mark 2}{(11.121,7.568)}
\gpcolor{color=gp lt color border}
\node[gp node left] at (8.777,7.106) {$h=10^{-7}$};
\gpcolor{rgb color={0.337,0.706,0.914}}
\gppoint{gp mark 3}{(1.320,8.441)}
\gppoint{gp mark 3}{(1.427,8.317)}
\gppoint{gp mark 3}{(1.535,8.195)}
\gppoint{gp mark 3}{(1.642,8.076)}
\gppoint{gp mark 3}{(1.749,7.958)}
\gppoint{gp mark 3}{(1.857,7.844)}
\gppoint{gp mark 3}{(1.964,7.731)}
\gppoint{gp mark 3}{(2.071,7.620)}
\gppoint{gp mark 3}{(2.179,7.512)}
\gppoint{gp mark 3}{(2.286,7.405)}
\gppoint{gp mark 3}{(2.393,7.301)}
\gppoint{gp mark 3}{(2.501,7.198)}
\gppoint{gp mark 3}{(2.608,7.098)}
\gppoint{gp mark 3}{(2.715,6.999)}
\gppoint{gp mark 3}{(2.823,6.901)}
\gppoint{gp mark 3}{(2.930,6.806)}
\gppoint{gp mark 3}{(3.037,6.712)}
\gppoint{gp mark 3}{(3.145,6.620)}
\gppoint{gp mark 3}{(3.252,6.529)}
\gppoint{gp mark 3}{(3.360,6.440)}
\gppoint{gp mark 3}{(3.467,6.352)}
\gppoint{gp mark 3}{(3.574,6.266)}
\gppoint{gp mark 3}{(3.682,6.182)}
\gppoint{gp mark 3}{(3.789,6.098)}
\gppoint{gp mark 3}{(3.896,6.016)}
\gppoint{gp mark 3}{(4.004,5.936)}
\gppoint{gp mark 3}{(4.111,5.856)}
\gppoint{gp mark 3}{(4.218,5.778)}
\gppoint{gp mark 3}{(4.326,5.701)}
\gppoint{gp mark 3}{(4.433,5.626)}
\gppoint{gp mark 3}{(4.540,5.551)}
\gppoint{gp mark 3}{(4.648,5.478)}
\gppoint{gp mark 3}{(4.755,5.405)}
\gppoint{gp mark 3}{(4.862,5.334)}
\gppoint{gp mark 3}{(4.970,5.264)}
\gppoint{gp mark 3}{(5.077,5.195)}
\gppoint{gp mark 3}{(5.184,5.127)}
\gppoint{gp mark 3}{(5.292,5.060)}
\gppoint{gp mark 3}{(5.399,4.994)}
\gppoint{gp mark 3}{(5.506,4.929)}
\gppoint{gp mark 3}{(5.614,4.865)}
\gppoint{gp mark 3}{(5.721,4.802)}
\gppoint{gp mark 3}{(5.828,4.739)}
\gppoint{gp mark 3}{(5.936,4.678)}
\gppoint{gp mark 3}{(6.043,4.617)}
\gppoint{gp mark 3}{(6.150,4.558)}
\gppoint{gp mark 3}{(6.258,4.499)}
\gppoint{gp mark 3}{(6.365,4.441)}
\gppoint{gp mark 3}{(6.472,4.383)}
\gppoint{gp mark 3}{(6.580,4.327)}
\gppoint{gp mark 3}{(6.687,4.271)}
\gppoint{gp mark 3}{(6.795,4.216)}
\gppoint{gp mark 3}{(6.902,4.162)}
\gppoint{gp mark 3}{(7.009,4.108)}
\gppoint{gp mark 3}{(7.117,4.055)}
\gppoint{gp mark 3}{(7.224,4.003)}
\gppoint{gp mark 3}{(7.331,3.951)}
\gppoint{gp mark 3}{(7.439,3.900)}
\gppoint{gp mark 3}{(7.546,3.850)}
\gppoint{gp mark 3}{(7.653,3.801)}
\gppoint{gp mark 3}{(7.761,3.752)}
\gppoint{gp mark 3}{(7.868,3.703)}
\gppoint{gp mark 3}{(7.975,3.656)}
\gppoint{gp mark 3}{(8.083,3.608)}
\gppoint{gp mark 3}{(8.190,3.562)}
\gppoint{gp mark 3}{(8.297,3.516)}
\gppoint{gp mark 3}{(8.405,3.470)}
\gppoint{gp mark 3}{(8.512,3.425)}
\gppoint{gp mark 3}{(8.619,3.381)}
\gppoint{gp mark 3}{(8.727,3.337)}
\gppoint{gp mark 3}{(8.834,3.294)}
\gppoint{gp mark 3}{(8.941,3.251)}
\gppoint{gp mark 3}{(9.049,3.209)}
\gppoint{gp mark 3}{(9.156,3.167)}
\gppoint{gp mark 3}{(9.263,3.126)}
\gppoint{gp mark 3}{(9.371,3.085)}
\gppoint{gp mark 3}{(9.478,3.044)}
\gppoint{gp mark 3}{(9.585,3.004)}
\gppoint{gp mark 3}{(9.693,2.965)}
\gppoint{gp mark 3}{(9.800,2.926)}
\gppoint{gp mark 3}{(9.907,2.887)}
\gppoint{gp mark 3}{(10.015,2.849)}
\gppoint{gp mark 3}{(10.122,2.811)}
\gppoint{gp mark 3}{(10.230,2.774)}
\gppoint{gp mark 3}{(10.337,2.737)}
\gppoint{gp mark 3}{(10.444,2.700)}
\gppoint{gp mark 3}{(10.552,2.664)}
\gppoint{gp mark 3}{(10.659,2.629)}
\gppoint{gp mark 3}{(10.766,2.593)}
\gppoint{gp mark 3}{(10.874,2.558)}
\gppoint{gp mark 3}{(10.981,2.524)}
\gppoint{gp mark 3}{(11.088,2.489)}
\gppoint{gp mark 3}{(11.196,2.455)}
\gppoint{gp mark 3}{(11.303,2.422)}
\gppoint{gp mark 3}{(11.410,2.389)}
\gppoint{gp mark 3}{(11.518,2.356)}
\gppoint{gp mark 3}{(11.625,2.323)}
\gppoint{gp mark 3}{(11.732,2.291)}
\gppoint{gp mark 3}{(11.840,2.259)}
\gppoint{gp mark 3}{(11.947,2.228)}
\gppoint{gp mark 3}{(11.121,7.106)}
\gpcolor{color=gp lt color border}
\node[gp node left] at (8.777,6.644) {$h=10^{-15}$};
\gpcolor{rgb color={0.902,0.624,0.000}}
\draw[gp path] (10.663,6.644)--(11.579,6.644);
\draw[gp path] (1.320,8.155)--(1.427,8.155)--(1.535,8.155)--(1.642,8.155)--(1.749,7.603)%
  --(1.857,8.155)--(1.964,8.155)--(2.071,7.603)--(2.179,7.603)--(2.286,7.603)--(2.393,7.603)%
  --(2.501,7.051)--(2.608,7.051)--(2.715,7.051)--(2.823,7.051)--(2.930,7.051)--(3.037,7.051)%
  --(3.145,6.500)--(3.252,6.500)--(3.360,5.948)--(3.467,6.500)--(3.574,5.948)--(3.682,6.500)%
  --(3.789,5.948)--(3.896,5.948)--(4.004,5.948)--(4.111,5.396)--(4.218,5.396)--(4.326,5.948)%
  --(4.433,5.948)--(4.540,5.396)--(4.648,5.948)--(4.755,5.396)--(4.862,5.396)--(4.970,5.396)%
  --(5.077,5.396)--(5.184,4.844)--(5.292,5.396)--(5.399,5.396)--(5.506,4.844)--(5.614,4.844)%
  --(5.721,4.844)--(5.828,4.844)--(5.936,4.292)--(6.043,4.292)--(6.150,4.844)--(6.258,4.292)%
  --(6.365,4.292)--(6.472,4.292)--(6.580,4.292)--(6.687,4.292)--(6.795,4.292)--(6.902,4.292)%
  --(7.009,4.292)--(7.117,3.740)--(7.224,4.292)--(7.331,3.740)--(7.439,3.740)--(7.546,3.740)%
  --(7.653,3.740)--(7.761,3.740)--(7.868,3.740)--(7.975,3.740)--(8.083,3.740)--(8.190,3.740)%
  --(8.297,3.188)--(8.405,3.188)--(8.512,3.188)--(8.619,3.188)--(8.727,3.188)--(8.834,3.188)%
  --(8.941,3.740)--(9.049,3.188)--(9.156,3.188)--(9.263,3.188)--(9.371,3.188)--(9.478,3.188)%
  --(9.585,3.188)--(9.693,3.188)--(9.800,3.188)--(9.907,3.188)--(10.015,2.637)--(10.122,2.637)%
  --(10.230,3.188)--(10.337,2.913)--(10.444,2.637)--(10.552,2.637)--(10.659,2.361)--(10.766,2.637)%
  --(10.874,2.637)--(10.981,2.361)--(11.088,2.637)--(11.196,2.361)--(11.303,2.637)--(11.410,2.361)%
  --(11.518,2.361)--(11.625,2.085)--(11.732,2.085)--(11.840,2.361)--(11.947,2.085);
\gppoint{gp mark 4}{(1.320,8.155)}
\gppoint{gp mark 4}{(1.427,8.155)}
\gppoint{gp mark 4}{(1.535,8.155)}
\gppoint{gp mark 4}{(1.642,8.155)}
\gppoint{gp mark 4}{(1.749,7.603)}
\gppoint{gp mark 4}{(1.857,8.155)}
\gppoint{gp mark 4}{(1.964,8.155)}
\gppoint{gp mark 4}{(2.071,7.603)}
\gppoint{gp mark 4}{(2.179,7.603)}
\gppoint{gp mark 4}{(2.286,7.603)}
\gppoint{gp mark 4}{(2.393,7.603)}
\gppoint{gp mark 4}{(2.501,7.051)}
\gppoint{gp mark 4}{(2.608,7.051)}
\gppoint{gp mark 4}{(2.715,7.051)}
\gppoint{gp mark 4}{(2.823,7.051)}
\gppoint{gp mark 4}{(2.930,7.051)}
\gppoint{gp mark 4}{(3.037,7.051)}
\gppoint{gp mark 4}{(3.145,6.500)}
\gppoint{gp mark 4}{(3.252,6.500)}
\gppoint{gp mark 4}{(3.360,5.948)}
\gppoint{gp mark 4}{(3.467,6.500)}
\gppoint{gp mark 4}{(3.574,5.948)}
\gppoint{gp mark 4}{(3.682,6.500)}
\gppoint{gp mark 4}{(3.789,5.948)}
\gppoint{gp mark 4}{(3.896,5.948)}
\gppoint{gp mark 4}{(4.004,5.948)}
\gppoint{gp mark 4}{(4.111,5.396)}
\gppoint{gp mark 4}{(4.218,5.396)}
\gppoint{gp mark 4}{(4.326,5.948)}
\gppoint{gp mark 4}{(4.433,5.948)}
\gppoint{gp mark 4}{(4.540,5.396)}
\gppoint{gp mark 4}{(4.648,5.948)}
\gppoint{gp mark 4}{(4.755,5.396)}
\gppoint{gp mark 4}{(4.862,5.396)}
\gppoint{gp mark 4}{(4.970,5.396)}
\gppoint{gp mark 4}{(5.077,5.396)}
\gppoint{gp mark 4}{(5.184,4.844)}
\gppoint{gp mark 4}{(5.292,5.396)}
\gppoint{gp mark 4}{(5.399,5.396)}
\gppoint{gp mark 4}{(5.506,4.844)}
\gppoint{gp mark 4}{(5.614,4.844)}
\gppoint{gp mark 4}{(5.721,4.844)}
\gppoint{gp mark 4}{(5.828,4.844)}
\gppoint{gp mark 4}{(5.936,4.292)}
\gppoint{gp mark 4}{(6.043,4.292)}
\gppoint{gp mark 4}{(6.150,4.844)}
\gppoint{gp mark 4}{(6.258,4.292)}
\gppoint{gp mark 4}{(6.365,4.292)}
\gppoint{gp mark 4}{(6.472,4.292)}
\gppoint{gp mark 4}{(6.580,4.292)}
\gppoint{gp mark 4}{(6.687,4.292)}
\gppoint{gp mark 4}{(6.795,4.292)}
\gppoint{gp mark 4}{(6.902,4.292)}
\gppoint{gp mark 4}{(7.009,4.292)}
\gppoint{gp mark 4}{(7.117,3.740)}
\gppoint{gp mark 4}{(7.224,4.292)}
\gppoint{gp mark 4}{(7.331,3.740)}
\gppoint{gp mark 4}{(7.439,3.740)}
\gppoint{gp mark 4}{(7.546,3.740)}
\gppoint{gp mark 4}{(7.653,3.740)}
\gppoint{gp mark 4}{(7.761,3.740)}
\gppoint{gp mark 4}{(7.868,3.740)}
\gppoint{gp mark 4}{(7.975,3.740)}
\gppoint{gp mark 4}{(8.083,3.740)}
\gppoint{gp mark 4}{(8.190,3.740)}
\gppoint{gp mark 4}{(8.297,3.188)}
\gppoint{gp mark 4}{(8.405,3.188)}
\gppoint{gp mark 4}{(8.512,3.188)}
\gppoint{gp mark 4}{(8.619,3.188)}
\gppoint{gp mark 4}{(8.727,3.188)}
\gppoint{gp mark 4}{(8.834,3.188)}
\gppoint{gp mark 4}{(8.941,3.740)}
\gppoint{gp mark 4}{(9.049,3.188)}
\gppoint{gp mark 4}{(9.156,3.188)}
\gppoint{gp mark 4}{(9.263,3.188)}
\gppoint{gp mark 4}{(9.371,3.188)}
\gppoint{gp mark 4}{(9.478,3.188)}
\gppoint{gp mark 4}{(9.585,3.188)}
\gppoint{gp mark 4}{(9.693,3.188)}
\gppoint{gp mark 4}{(9.800,3.188)}
\gppoint{gp mark 4}{(9.907,3.188)}
\gppoint{gp mark 4}{(10.015,2.637)}
\gppoint{gp mark 4}{(10.122,2.637)}
\gppoint{gp mark 4}{(10.230,3.188)}
\gppoint{gp mark 4}{(10.337,2.913)}
\gppoint{gp mark 4}{(10.444,2.637)}
\gppoint{gp mark 4}{(10.552,2.637)}
\gppoint{gp mark 4}{(10.659,2.361)}
\gppoint{gp mark 4}{(10.766,2.637)}
\gppoint{gp mark 4}{(10.874,2.637)}
\gppoint{gp mark 4}{(10.981,2.361)}
\gppoint{gp mark 4}{(11.088,2.637)}
\gppoint{gp mark 4}{(11.196,2.361)}
\gppoint{gp mark 4}{(11.303,2.637)}
\gppoint{gp mark 4}{(11.410,2.361)}
\gppoint{gp mark 4}{(11.518,2.361)}
\gppoint{gp mark 4}{(11.625,2.085)}
\gppoint{gp mark 4}{(11.732,2.085)}
\gppoint{gp mark 4}{(11.840,2.361)}
\gppoint{gp mark 4}{(11.947,2.085)}
\gppoint{gp mark 4}{(11.121,6.644)}
\gpcolor{color=gp lt color border}
\draw[gp path] (1.320,8.441)--(1.320,0.985)--(11.947,0.985)--(11.947,8.441)--cycle;
%% coordinates of the plot area
\gpdefrectangularnode{gp plot 1}{\pgfpoint{1.320cm}{0.985cm}}{\pgfpoint{11.947cm}{8.441cm}}
\end{tikzpicture}

        \caption{\textbf{Accuracy of the finite difference method.} The accuracy
        and stability of the approximate derivative actually gets worse with 
        decreasing step size. The optimal step size for this case is 
        $h=10^{-7}$.}
        \label{fig:fd-accuracy}
    \end{figure}
    
    The method of automatic differentiation, sometimes also referred to as 
    algorithmic differentiation, addresses the weaknesses of the finite 
    difference method by providing a systematic way to calculate derivatives 
    numerically to arbitrary precision. The goal of \texttt{PackageName} is to 
    implement the forward mode of automatic differentiation, as it is a relevant
    feature that even some mainstream machine learning libraries, such as 
    PyTorch, lack. 
    \begin{itemize}
        \item Introduce numerical methods for calculating derivatives (e.g. 
        finite difference) and discuss their strengths and weaknesses
        \item Propose automatic differentiation (AD) as a solution
        \item Explain the need for implementing the forward mode of AD
        \item What are the advantages of forward mode compared to reverse mode?
    \end{itemize}
    \section*{Background}
    
    \section*{Usage Guide}
    Assuming that the user already has the latest version of Python installed, 
    the first step towards using \texttt{PackageName} will be the installation, 
    which can be accomplished using \texttt{conda} or \texttt{pip}. This process
    will vary slightly, depending on the operating system, but essentially, the 
    user will execute commands from the terminal to update the package manager 
    and install \texttt{PackageName}. 
    
    Once \texttt{PackageName} is installed, the user must import it to be able 
    to use it. Here, the user will have the choice to either import the entire 
    library or to choose only a subset of modules, classes, or methods to import.
    For instance, importing \texttt{PackageName.linear} will only import the 
    automatic differentiation class, and all of its methods, for linear 
    functions of the form $f = \alpha x + \beta$. Then, instances of 
    \texttt{PackageName.linear} may be defined by passing in the relevant 
    arguments. For example, defining \texttt{f = PackageName.linear(2.0)} would 
    instantiate an object whose function value is \texttt{2.0} and whose 
    derivative is the default value (i.e. \texttt{da = 1.0}). However, the user 
    will also have the ability to define a value for the derivative, if 
    necessary (e.g. \texttt{g = PackageName.linear(3.0, 2.1)}). Instances of 
    other classes pertaining to \texttt{PackageName} would be defined in a 
    similar manner. 
    
    Once the user defines an instance of a class, they may perform standard 
    elementary operations on that object. For example, \texttt{f*g} would return
    a new object whose function value is the product of the function values of 
    \texttt{f} and \texttt{g} and whose derivative value is the product of the
    corresponding derivative values of \texttt{f} and \texttt{g}, according 
    to the product rule. 
    \begin{equation}
        \left.\frac{d}{dx}(fg)\right|_{x=a} 
        = 
        \left.\vphantom{\frac{d}{dx}}f'(x)g(x) + f(x)g'(x)\right|_{x=a}
    \end{equation}
    This way, the user would have the ability to calculate derivatives of 
    composite functions using elementary operations while retaining arbitrary 
    precision at each evaluation step.
    \begin{itemize}
        \item Should this be generalized?
        \item What about other elementary functions? Should examples of these be
        included as well?
        \item Is pseudo-code really necessary here?
    \end{itemize} 
\end{document}
